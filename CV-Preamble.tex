%%%%%%%%%%%%%%%%%%%%%%%%%%%%%%%%%%%%%%%%%%%%%%%%%%%%%%%%%%%%%%%%%
%% SIMPLE-RESUME-CV
%% <https://github.com/zachscrivena/simple-resume-cv>
%% This is free and unencumbered software released into the
%% public domain; see <http://unlicense.org> for details.
%%%%%%%%%%%%%%%%%%%%%%%%%%%%%%%%%%%%%%%%%%%%%%%%%%%%%%%%%%%%%%%%%

% Long table for page layout.
\usepackage{longtable}

% Geometry package for page margins.
\usepackage[
left=0.60in,	% was 0.70in
right=0.60in,	% was 0.70in
top=0.35in,		% was 0.70in
bottom=0.00in,
nohead,
includefoot]{geometry}

% Hyphenation: Disabled.
\usepackage[none]{hyphenat}

\usepackage{setspace} % Used for changing line spacing properties

%\usepackage{showframe}	% Render a frame marking the margins of the current document

\usepackage{enumitem}
\setlist[itemize]{leftmargin=*}

% XeLaTeX packages.
\usepackage{fontspec}
\defaultfontfeatures{Ligatures=TeX}
\usepackage{xunicode}
\usepackage{xltxtra}

\def\Csharp{C\#}

\newcommand{\ts}{\textsuperscript}
%\usepackage[super]{nth}

% Font: Use "Tinos" as the main typeface (\textnormal{}, \normalfont).
% The "Tinos" fonts are released under the Apache License Version 2.0,
% and can be downloaded for free at <http://www.fontsquirrel.com/fonts/tinos>.
% Symbol table: <http://www.fileformat.info/info/unicode/font/tinos/grid.htm>
\setmainfont
[Path=./Fonts/Tinos/,
ItalicFont=Tinos-Italic,
BoldFont=Tinos-Bold,
BoldItalicFont=Tinos-BoldItalic]
{Tinos-Regular.ttf}

% Sans-serif font: Switched to "Tinos".
\renewcommand{\sffamily}{\rmfamily}

% Typewriter (monospace) font: Switched to "Tinos".
\renewcommand{\ttfamily}{\rmfamily}

% Small caps font: Switched to "Tinos".
\renewcommand{\scshape}{\rmfamily}

% Secondary font: "GNU FreeFont".
% The "GNU FreeFont" fonts are released under the
% GNU General Public License Version 3, and can be downloaded
% for free at <https://savannah.gnu.org/projects/freefont/>.
\newcommand{\UseSecondaryFont}{\fontspec
[Path=./Fonts/GNUFreeFont/,
ItalicFont=FreeSerifItalic,
BoldFont=FreeSerifBold,
BoldItalicFont=FreeSerifBoldItalic]
{FreeSerif.otf}}

% Symbols (unicode).
\newcommand{\BulletSymbol}{{\char"2022}}
\newcommand{\TildeSymbol}{{\char"007E}}

% PDF settings and properties.
\usepackage{hyperref}

% Headers and footers: Blank header, page number in footer.
\makeatletter
\def\ps@plain{%
\def\@oddhead{}%
\def\@evenhead{}%
%\def\@oddfoot{\footnotesize\hfill{Page}~{\thepage}~of~\pageref{LastPage}\hfill}%
%\def\@evenfoot{\footnotesize\hfill{Page}~{\thepage}~of~\pageref{LastPage}\hfill}
}
\makeatother

\pagestyle{empty} % was \pagestyle{plain}
%\pagenumbering{gobble}

% Paragraph style: No indentation.
\setlength{\parindent}{0in}

% Footnotes: Use symbols instead of numbers for labels.
\renewcommand{\thefootnote}{\fnsymbol{footnote}}

% Current date and time.
%\usepackage[yyyymmdd,24hr]{datetime}
\usepackage{datetime}
\renewcommand{\dateseparator}{.}
% {\today}~{\currenttime}

% Abbreviations for months.
\newcommand{\LongMonth}[1]{%
\ifcase#1\relax
\or January%
\or February%
\or March%
\or April%
\or May%
\or June%
\or July%
\or August%
\or September%
\or October%
\or November%
\or December%
\fi}
\newcommand{\ShortMonth}[1]{%
\ifcase#1\relax
\or Jan%
\or Feb%
\or Mar%
\or Apr%
\or May%
\or Jun%
\or Jul%
\or Aug%
\or Sep%
\or Oct%
\or Nov%
\or Dec%
\fi}

% Select datestamp format.
\def\DatestampFormatSelection{4}	% was 2

% Datestamp format: {yyyy}{MM}{dd} ---> yyyy-MM-dd (e.g., 2010-12-31).
\ifnum\DatestampFormatSelection=1
\newcommand{\DatestampYMD}[3]{\mbox{#1-#2-#3}}
\newcommand{\DatestampYM}[2]{\mbox{#1-#2}}
\newcommand{\DatestampY}[1]{#1}
\fi

% Datestamp format: {yyyy}{MM}{dd} ---> MMM yyyy (e.g., Dec 2010).
\ifnum\DatestampFormatSelection=2
\newcommand{\DatestampYMD}[3]{\mbox{\ShortMonth{#2} #1}}
\newcommand{\DatestampYM}[2]{\mbox{\ShortMonth{#2} #1}}
\newcommand{\DatestampY}[1]{#1}
\fi

% Datestamp format: {yyyy}{MM}{dd} ---> MMMM yyyy (e.g., December 2010).
\ifnum\DatestampFormatSelection=3
\newcommand{\DatestampYMD}[3]{\mbox{\LongMonth{#2} #1}}
\newcommand{\DatestampYM}[2]{\mbox{\LongMonth{#2} #1}}
\newcommand{\DatestampY}[1]{#1}
\fi

% Datestamp format: {yyyy}{MM}{dd} ---> yyyy (e.g., 2010).
\ifnum\DatestampFormatSelection=4
\newcommand{\DatestampYMD}[3]{\fontsize{9}{10.3}\selectfont#1\normalsize}
\newcommand{\DatestampYM}[2]{\fontsize{9}{10.3}\selectfont#1\normalsize}
\newcommand{\DatestampY}[1]{\fontsize{9}{10.3}\selectfont#1\normalsize}
\fi

% New datestamp format: {yyyy}{MM}{dd} ---> MMM yyyy (e.g., Dec 2010).
\newcommand{\DatestampYMDnew}[3]{\fontsize{9}{10.3}\selectfont\mbox{\ShortMonth{#2} #1}\normalsize}
\newcommand{\DatestampYMnew}[2]{\fontsize{9}{10.3}\selectfont\mbox{\ShortMonth{#2} #1}\normalsize}
\newcommand{\DatestampYnew}[1]{\fontsize{9}{10.3}\selectfont#1\normalsize}

% Present

\newcommand{\PresentDate}{\fontsize{9}{10.3}\selectfont Present\normalsize}

% Macro: title (name).
\renewcommand{\title}[1]{%
\pdfbookmark[1]{#1}{#1}%
\par\begin{center}%
\par\begin{LARGE}%
\textbf{#1}%
\par\end{LARGE}%
\par\end{center}%
\par\vspace{-2.0em}\par} % was -1.75em

% Macro: subtitle (personal information below name).
\newenvironment{subtitle}
{\par\begin{center}%
\par\begin{footnotesize}\onehalfspacing}
{\par\end{footnotesize}%
\par\end{center}\par}

% Macro: body (rest of the document).
\newenvironment{body}
{\par\vspace{-2.1em}\par		% was -1em
\begin{longtable}[c]{@{\hspace*{-0pt}}p{0.13\textwidth}p{0.85\textwidth}}}	% Also works without the asterisk
%\begin{longtable}[c]{@{\hspace*{-10pt}}p{0.15\textwidth}p{0.83\textwidth}}}
{\par\end{longtable}\par}

% Macro: section (new section for Education, Research Experience, etc.).
\renewcommand{\section}[3]{\\[-17pt]\pdfbookmark[2]{#2}{#3}\\%
{\fontsize{11pt}{11pt}\selectfont\bfseries\raggedright%
#1}&}

% Macro: subsection.
\renewcommand{\subsection}[3]{\vspace{-14pt}% % was \par~\vskip-\baselineskip
\pdfbookmark[3]{#2}{#3}\par%
{\fontsize{9pt}{9.6pt}\selectfont\bfseries\raggedright%
#1}%	% was \MakeUppercase{#1}
\vspace{0.1em}}

% Macro: BigEntryGap (big vertical gap between entries in the same section).
\newcommand{\BigEntryGap}{\\[-0.7em]~&}	% was -0.5em
\newcommand{\BigEntryGapNoBreak}{\par\vspace{0.4em}\par}	% was 0.7em

% Macro: EntryGap (vertical gap between entries in the same section).
\newcommand{\EntryGap}{\\[-1em]~&}	% was -0.8em
\newcommand{\EntryGapNoBreak}{\par\vspace{0.1em}\par}	% was 0.4em

% Macro: itemslist (Used to describe headings that don't use details)
\newenvironment{itemslist}
{\vspace*{-2.1em}\par\begingroup\fontsize{10}{10.3}\selectfont\begin{itemize}\setlength{\itemsep}{3pt}\setlength{\parskip}{0cm}}
{\end{itemize}\par\endgroup\par\vspace{-1.1em}}

% Macro: detail (text in smaller font under an entry).
\newenvironment{detail}
{\vspace*{-0.8em}\par\setlength{\rightskip}{2cm}\noindent\begingroup\fontsize{9}{10.5}\selectfont}
{\par\endgroup\par}

% Macro: detail with no sub-heading (text in smaller font under main heading)
\newenvironment{detailwithoutsubheading}
{\par\setlength{\rightskip}{2cm}\begingroup\fontsize{9}{10.5}\selectfont}
{\par\endgroup\par}

% Macro: BulletItem.
\newsavebox{\BulletItemIndentation}
\newlength{\BulletItemIndentationWidth}
\newcommand{\BulletItem}{\par%
\savebox{\BulletItemIndentation}{\hspace{0.8em}\BulletSymbol\hspace{0.4em}}%
\settowidth{\BulletItemIndentationWidth}{\usebox{\BulletItemIndentation}}%
\noindent\hangafter=1\hangindent=\BulletItemIndentationWidth\ignorespaces%
\usebox{\BulletItemIndentation}\ignorespaces}

% Macro: NumberedItem.
\newsavebox{\NumberedItemIndentation}
\newlength{\NumberedItemIndentationWidth}
\newcommand{\NumberedItem}[1]{\par%
\savebox{\NumberedItemIndentation}{{#1}\hspace{0.4em}}%
\settowidth{\NumberedItemIndentationWidth}{\usebox{\NumberedItemIndentation}}%
\noindent\hangafter=1\hangindent=\NumberedItemIndentationWidth\ignorespaces%
\usebox{\NumberedItemIndentation}\ignorespaces}

% Macro: CharSpace (for aligning shorter numbers).
\newlength{\CharWidth}
\newcommand{\CharSpace}{\settowidth{\CharWidth}{8}\hspace{\CharWidth}}

% Macro: hide.
\newcommand{\hide}[1]{}

% Symbols/icons
%\usepackage{marvosym}
\usepackage{fontawesome}      % Reqiures the fontspec package when using the XeLaTeX or LuaLaTeX engine